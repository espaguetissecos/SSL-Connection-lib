monta un servidor T\-C\-P con conexion S\-S\-L\hypertarget{unknown_case_SYNOPSIS}{}\section{S\-Y\-N\-O\-P\-S\-I\-S}\label{unknown_case_SYNOPSIS}
{\bfseries \#include} {\bfseries \char`\"{}\-S\-S\-Llib.\-h\char`\"{}} 

{\bfseries S\-S\-L$\ast$} {\bfseries aceptar\-\_\-canal\-\_\-seguro\-\_\-\-S\-S\-L} {\bfseries }({\bfseries S\-S\-L\-\_\-\-C\-T\-X$\ast$} ctx, {\bfseries cint} sockfd, {\bfseries int} puerto, {\bfseries int} tam, {\bfseries struct} sockaddr\-\_\-in ip4addr{\bfseries })\hypertarget{unknown_case_descripcion}{}\section{D\-E\-S\-C\-R\-I\-P\-C\-IÓ\-N}\label{unknown_case_descripcion}
Dado un contexto S\-S\-L y un descriptor de socket esta función se encargará de las llamadas a bind, listen y accept, todo ello siguiendo el prot. S\-S\-L\hypertarget{cerrar_canal_SSL_return}{}\section{R\-E\-T\-U\-R\-N}\label{cerrar_canal_SSL_return}
Devuelve N\-U\-L\-L en caso de error o la estructura S\-S\-L$\ast$ inicializada en caso de exito.\hypertarget{unknown_case_seealso}{}\section{V\-E\-R T\-A\-M\-B\-IÉ\-N}\label{unknown_case_seealso}
{\bfseries inicializar\-\_\-nivel\-\_\-\-S\-S\-L(3)}, {\bfseries fijar\-\_\-contexto\-\_\-\-S\-S\-L(3)}, (3), {\bfseries evaluar\-\_\-post\-\_\-connectar\-\_\-\-S\-S\-L(3)}, {\bfseries enviar\-\_\-datos\-\_\-\-S\-S\-L(3)}, {\bfseries recibir\-\_\-datos\-\_\-\-S\-S\-L(3)}, {\bfseries cerrar\-\_\-canal\-\_\-\-S\-S\-L(3)} {\bfseries }  authors A\-U\-T\-O\-R Francisco Andreu Sanz (\href{mailto:francisco.andreu@estudiante.uam.es}{\tt francisco.\-andreu@estudiante.\-uam.\-es}) Javier Martinez Hernandez (\href{mailto:javier.martinez@estudiante.uam.es}{\tt javier.\-martinez@estudiante.\-uam.\-es}) 