Muestra información sobre un nick o canal.\hypertarget{unknown_case_SYNOPSIS}{}\section{S\-Y\-N\-O\-P\-S\-I\-S}\label{unknown_case_SYNOPSIS}
{\bfseries \#include} {\bfseries $<$I\-R\-Ccommands.\-h$>$} 

{\bfseries void} {\bfseries who\-\_\-case} {\bfseries }({\bfseries char{\bfseries $\ast${\bfseries prefix{\bfseries },} {\bfseries char{\bfseries $\ast${\bfseries mask{\bfseries },} {\bfseries char{\bfseries $\ast${\bfseries oppar{\bfseries },} {\bfseries char{\bfseries $\ast$$\ast${\bfseries nick{\bfseries },} const{\bfseries int{\bfseries sd{\bfseries })}  } } descripcion} D\-E\-S\-C\-R\-I\-P\-C\-IÓ\-N}  Muestra} información} sobre} un} nick} de manera mas escueta. En el caso de poner un \#canal mostrará a los usuarios que esten dentro. Se permite recibir un mensaje como información extra.

Recibe como parámetros el prefijo que nos da la funcion de parseo del comando, mask indica el nick o canal del cual queremos información, un mensaje con información extra, el nick del usuario que envia el mensaje y su socket correspondiente.\hypertarget{unknown_case_retorno}{}\section{R\-E\-T\-O\-R\-N\-O}\label{unknown_case_retorno}
No devuelve nada.\hypertarget{unknown_case_authors}{}\section{A\-U\-T\-O\-R}\label{unknown_case_authors}
Francisco Andreu Sanz (\href{mailto:francisco.andreu@estudiante.uam.es}{\tt francisco.\-andreu@estudiante.\-uam.\-es}) Javier Martínez Hernández (\href{mailto:javier.maritnez@estudiante.uam.es}{\tt javier.\-maritnez@estudiante.\-uam.\-es}) 