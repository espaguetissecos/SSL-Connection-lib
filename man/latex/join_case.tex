Añade un cliente a un canal.\hypertarget{unknown_case_SYNOPSIS}{}\section{S\-Y\-N\-O\-P\-S\-I\-S}\label{unknown_case_SYNOPSIS}
{\bfseries \#include} {\bfseries $<$I\-R\-Ccommands.\-h$>$} 

{\bfseries void} {\bfseries join\-\_\-case} {\bfseries }({\bfseries char{\bfseries $\ast${\bfseries prefix{\bfseries },} {\bfseries char{\bfseries $\ast${\bfseries newnick{\bfseries },} {\bfseries char{\bfseries $\ast${\bfseries msg{\bfseries },} {\bfseries char{\bfseries $\ast$$\ast${\bfseries nick{\bfseries },} const{\bfseries int{\bfseries sd{\bfseries })}  } } descripcion} D\-E\-S\-C\-R\-I\-P\-C\-IÓ\-N}  Añade} un} cliente} a} un} canal. Si este no existe lo crea. Se puede incluir contraseña para la creación, esta será necesaria para el acceso a este canal.

Recibe como parámetros el prefijo que nos da la funcion de parseo del comando, canal que queremos crear o unirnos, contraseña para el canal, un posible mensaje, el nick del usuario que envia el mensaje, su socket correspondiente y la cadena del comando que recibimos.\hypertarget{unknown_case_retorno}{}\section{R\-E\-T\-O\-R\-N\-O}\label{unknown_case_retorno}
No devuelve nada.\hypertarget{unknown_case_authors}{}\section{A\-U\-T\-O\-R}\label{unknown_case_authors}
Francisco Andreu Sanz (\href{mailto:francisco.andreu@estudiante.uam.es}{\tt francisco.\-andreu@estudiante.\-uam.\-es}) Javier Martínez Hernández (\href{mailto:javier.maritnez@estudiante.uam.es}{\tt javier.\-maritnez@estudiante.\-uam.\-es}) 