Muestra una lista con los canales en el servidor.\hypertarget{unknown_case_SYNOPSIS}{}\section{S\-Y\-N\-O\-P\-S\-I\-S}\label{unknown_case_SYNOPSIS}
{\bfseries \#include} {\bfseries $<$I\-R\-Ccommands.\-h$>$} 

{\bfseries void} {\bfseries list\-\_\-case} {\bfseries }({\bfseries char{\bfseries $\ast${\bfseries prefix{\bfseries },} {\bfseries char{\bfseries $\ast${\bfseries channel{\bfseries },} {\bfseries char{\bfseries $\ast${\bfseries target{\bfseries },} {\bfseries char{\bfseries $\ast$$\ast${\bfseries nick{\bfseries },} const{\bfseries int{\bfseries sd{\bfseries })}  } } descripcion} D\-E\-S\-C\-R\-I\-P\-C\-IÓ\-N}  Muestra} una} lista} con} los} canales en el servidor. Tambien se puede especificar el servidor.

Recibe como parámetros el prefijo que nos da la funcion de parseo del comando, el canal sobre el cual queremos la información, target nos indica un servidor especifico para solicitar la información, el nick del usuario que envia el mensaje y su socket correspondiente.\hypertarget{unknown_case_retorno}{}\section{R\-E\-T\-O\-R\-N\-O}\label{unknown_case_retorno}
No devuelve nada.\hypertarget{unknown_case_authors}{}\section{A\-U\-T\-O\-R}\label{unknown_case_authors}
Francisco Andreu Sanz (\href{mailto:francisco.andreu@estudiante.uam.es}{\tt francisco.\-andreu@estudiante.\-uam.\-es}) Javier Martínez Hernández (\href{mailto:javier.maritnez@estudiante.uam.es}{\tt javier.\-maritnez@estudiante.\-uam.\-es}) 