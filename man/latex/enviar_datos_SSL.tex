Envia datos de forma segura con S\-S\-L\hypertarget{unknown_case_SYNOPSIS}{}\section{S\-Y\-N\-O\-P\-S\-I\-S}\label{unknown_case_SYNOPSIS}
{\bfseries \#include} {\bfseries \char`\"{}\-S\-S\-Llib.\-h\char`\"{}} 

{\bfseries int} {\bfseries enviar\-\_\-datos\-\_\-\-S\-S\-L} {\bfseries }({\bfseries S\-S\-L$\ast$} ssl, {\bfseries char$\ast$} buf, {\bfseries const} char$\ast$ prv\-Key\-File, {\bfseries const} char$\ast$ cert\-File{\bfseries })\hypertarget{unknown_case_descripcion}{}\section{D\-E\-S\-C\-R\-I\-P\-C\-IÓ\-N}\label{unknown_case_descripcion}
Esta función será el equivalente a la función de envío de mensajes que se realizó en la práctica 1(Serv.\-I\-R\-C) o 2, pero será utilizada para enviar datos a través del canal seguro. Es importante que sea genérica y pueda ser utilizada independientemente de los datos que se vayan a enviar.\hypertarget{cerrar_canal_SSL_return}{}\section{R\-E\-T\-U\-R\-N}\label{cerrar_canal_SSL_return}
Devuelve 0 en caso de error o 1 en caso de exito .\hypertarget{unknown_case_seealso}{}\section{V\-E\-R T\-A\-M\-B\-IÉ\-N}\label{unknown_case_seealso}
{\bfseries inicializar\-\_\-nivel\-\_\-\-S\-S\-L(3)}, {\bfseries conectar\-\_\-canal\-\_\-seguro\-\_\-\-S\-S\-L(3)}, {\bfseries aceptar\-\_\-canal\-\_\-seguro\-\_\-\-S\-S\-L(3)}, {\bfseries evaluar\-\_\-post\-\_\-connectar\-\_\-\-S\-S\-L(3)}, {\bfseries enviar\-\_\-datos\-\_\-\-S\-S\-L(3)}, {\bfseries recibir\-\_\-datos\-\_\-\-S\-S\-L(3)}, {\bfseries cerrar\-\_\-canal\-\_\-\-S\-S\-L(3)} {\bfseries }  authors A\-U\-T\-O\-R Francisco Andreu Sanz (\href{mailto:francisco.andreu@estudiante.uam.es}{\tt francisco.\-andreu@estudiante.\-uam.\-es}) Javier Martinez Hernandez (\href{mailto:javier.martinez@estudiante.uam.es}{\tt javier.\-martinez@estudiante.\-uam.\-es}) 